\section{Conclusions}\label{sec:conclusions}

Now we can see the results obtained from the adoption of the techniques we have studied in the previous sections. The uniflow pattern allowed us to identify a consistent and global state which describes the state of the project and the business logic. Moreover, in conjunction with the adoption of a serverless architecture, it led to a slim architecture for the collaboration. Using the uniflow pattern with the Virtual DOM, permitted us to automatically render only changes to the state without complex optimization algorithms, like the one based on differences and patches used for the 3D rendering of the building model. In addition the system is extensible with the addition of geometric elements, in fact the user can register external services to add new ones. In conclusion, use of the Web Components allow us to expose our software in a modular way, improving maintainability.

\subsection{Future work}

The system could be expanded thanks to the modularity given by the serverless paradigm. As we already have a good user experience, to help the adoption of this application in other modeling contexts, we should improve the catalog adding new building elements