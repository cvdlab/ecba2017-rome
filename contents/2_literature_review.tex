\section{Literature review}

Come visto nella sezione precedente, questo lavoro ha lo scopo di produrre un ambiente di progettazione architettonica per il web. Evitando il confronto con ``desktop apps'', perch\`e molto pi\`u complete (anche se poco portabili) esistono diversi progetti correlati su piattaforma web. Alcuni si occupano solo di visualizzazione come la parte web di \textbf{shapespark} (esiste anche un plugin desktop per la modellazione), altri offrono anche un servizio di modellazione come \textbf{bak3d} (vedi~\cite{Spini:2016:WIA:2945292.2945309}). Tuttavia la modalit\`a di interazione offerta da questo \`e particolarmente complicata. Un altro prodotto molto complesso \`e \textbf{playcanvas}, che essendo un game engine completo, offre anche un motore fisico ed altre funzionalit\`a ai modelli realizzati. Rimanendo all'ambito architettonico, un progetto interessante \`e \textbf{floorplan} di autodesk (http://www.homestyler.com/floorplan/).\\
Tuttavia il nostro obiettivo era anche quello di esplorare il paradigma dei web components e del pattern uniflow. Il risultato atteso \`e quello di riuscire a realizzare non solo una completa infrastruttura per il disegno ma anche una piattaforma estendibile da parte degli utenti mediante l'introduzione di nuovi componenti. In effetti questa \`e la vera e propria differenza con gli strumenti citati precedentemente, perché l'architettura proposta non si limita a risolvere il problema della modellazione ma offre uno strumento di produzione di modellatori architettonici.