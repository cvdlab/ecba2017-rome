\section{Application Experience}\label{sec:application}

The application experience focus on supporting the user in a building modeling task. The exploited modeling approach requires the user to face as much as possible a bidimensional interface which allows her to define the floorplan and to place complex architectural elements (here called \emph{building elements}) on it. Such \emph{building elements} can be found in a pre-filled catalog, and when required can be further configured and customized through a side panel. This modeling approach move part of the complexity toward the developer of the customizable building elements, leaving to the final user the task to place and to configure the employed elements. A rich catalog of elements is thus crucial to answer to the users' modeling requirements.

Once the floorplan has been defined according to the \emph{place-and-configure} approach, the system can automatically generate a 3D model which can be explored externally or in first person view, as shown in Figure~\ref{FIGURA_IN_PRIMA_PERSONA}. Each  \emph{building element} in fact comprises either a \emph{2D generating function} than a \emph{3D generating function}, used to obtain models respectively used in the 2D floorplan definition and in 3D generated model.

The tool also has support for layers the user can exploit to organize her project, for example to group together semantically homogenous elements.

\subsection{Building Elements Catalog}\label{ssec:elements}

The catalog comprises four types of elements, grouped according to both inherent characteristics and user interaction needed to add it to the project and configure it.

\begin{itemize}
\item \emph{Lines}. An element which belongs to this category is drawn selecting a start point and an end point. To move this element one can drags one of the end point or can drags the entire line. An example: a \emph{wall}.

\item \emph{Openings}. Each opening is an element that is linked to an element of the \emph{line} type, making an hole on it. The user create a new opening dragging it on a chosen line. Examples are \emph{doors} and \emph{windows}.

\item \emph{Areas}. Each area is an element which is automatically generated from walls. An example: the \emph{basement}.

\item \emph{Objects}. Each element which is freely inserted into space with a drag and drop interaction is an object. Examples are tables and chairs.
\end{itemize}

\subsection{User Interface}\label{ssec:ui}

Figure~\ref{fig2D} show the application user interface. It consists of the following components: 

\textbf{Toolbar:} it contains the button buttons mapping all operations the user can do in that context

\textbf{Sidebar:} it contains all elements in the current layer, the layer list and the properties for the selected building element

\textbf{Content area}: it supports all the 2D interactions for the user. Inside this area we can see the current plan with placeholders for all building elements\\

Figure~\ref{figCatalogo} shows the building element catalog. When the user select one of the boxes, the application returns to the 2D area and starts the interaction for the chosen building element.

Figures~\ref{fig3D},~\ref{figPalazzo},~\ref{figAltro3D} show some examples of 3D visualization.
