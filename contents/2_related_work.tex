\section{Related work}\label{sec:related_work}

In this section we highlight some remarkable experiences aligned with the aim of our project. There are plenty of Desktop applications worth to be mentioned and analyzed, but in the following we deliberately focus on Web based works. 

\emph{Shapespark}\footnote{\url{https://www.shapespark.com/}} offers a web viewer of remarkable quality that allow the user to move inside a virtual 3D indoor environment. Modeling phase is served in the form of plugins for different Desktop proprietary solutions.

\emph{Playcanvas}\footnote{\url{https://playcanvas.com/}} is a complete and powerful web based game creation platform which offers an integrated physical engine and a whole set of functionalities to support modeling. Although powerful and relatively simple to use, it doesn't focus on buildings modeling.

\emph{Floorplan}\footnote{\url{http://www.homestyler.com/floorplan/}} has been developed by Autodesk specifically for the architectural field, and for indoor renewal projects in particular. It is a 2D modeling tool which offer also a 3D walk-through mode.

Spini et al. \cite{Spini:2016:WIA:2945292.2945309} introduced a Web modeling and baking service for indoor environments. The modeling tools exposes a 3D interaction the user may not be accustomed to, an hitch we tried to outflank by avoiding 3D modeling interaction and let the user only face a ``metaphoric" 2D interface.

As regards support for users collaboration it worths to be mentioned the \emph{Operational Transformation} (OT) approach~\cite{Ellis:1989:CCG:66926.66963}: a group of nodes exchanges messages without a central control point. Two main properties hold in this setup: (i) changes are relative to other user's changes (it works on ``diffs'') and (ii) no matter in which order concurrent changes are applied, the final document is the same. In our serverless architecture however, external (third parties) central synchronization point are allowed, making complexity introduced by protocols like OT less effective.