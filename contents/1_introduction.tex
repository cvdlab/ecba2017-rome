\section{Introduction}

This work intend to show a web based tool for building design. About the architectural point has been introduced a serverless architecture (see~\cite{Roberts}). This kind of architecture uses third party services (typically in the cloud) or functions executed into ephemeral containers (may only last for one invocation) to manage the internal state and for server-side logic.

This tool aims to build building model thought a simplified UI (User Interface) that avoid complex 3D interactions that are replaced by interactions with two-dimensional symbolic part of the building.

The web character of the project permitted to test API for remote users collaboration, that allow to build together a building also if far each other.

The tool offer an high level of customization, thanks to: (i) \textbf{Web Components pattern} that allow to add new features; (ii) a customizable \textbf{catalog}, that allow to adapt building part to the specific context in which the tool is using; (iii) a dynamic metadata handler the allow to set info on each part of the building.

Next sections describe this concepts, showing how they are realized and which choose were made:
 In details:

In the section~\ref{sec:literature} we will show some related works about web building modelling with a focus on strengths and weaknesses of our work.\\\\
In the section ~\ref{sec:methodology} we will show the core of the work introducing and commenting the most important architectural choose that we done\\\\
In the section ~\ref{sec:results} we will show the solution of the methodical choose that were made, plus a focus on some future works related to the project.\\\\



Scopo di questo lavoro \`e quello di presentare uno strumento di progettazione architettonica basato su web. Dal punto di vista architetturale, \`e stata introdotta una serverless architecture (vedi~\cite{Roberts}). Un'architettura di questo tipo si basa massivamente su servizi di terze parti (typically in the cloud) o su funzioni invocate all'interno di ephemeral containers (may only last for one invocation) per la gestione dello stato e della logica server-side.

Questo strumento si propone di realizzare modelli di edifici attraverso l'introduzione di un'interfaccia semplificata che mira ad evitare complicate interazioni con i modelli tridimensionali cercando invece di rimpiazzarle con interazioni su rappresentazioni bidimensionali simboliche dell'edificio.

La natura web ha permesso di sperimentare API per la collaborazione tra utenti remoti, che possono cos\`i progettare un edificio contemporaneamente su pc diversi.

Molta cura \`e stata poi posta sull'espandibilit\`a del sistema, dando agli sviluppatori un elevato livello di personalizzazione. Ci\`o \`e possibile: (i) Sfruttando \textbf{web components pattern} per aggiungere nuove funzionalità; (ii) introducendo un \textbf{catalogo} personalizzabile, in grado di fornire elementi architetturali specifici del contesto in cui l'applicazione viene utilizzata; (iii) introducendo un gestore dinamico di \textbf{metadati}, in grado di 
attribuire informazioni su ogni componente dell'edificio.

Le sezioni che seguono mostrano nel dettaglio i concetti espressi qui, mostrandone la realizzazione e ponendo l'accento sugli aspetti architetturali. In particolare:\\\\
Nella sezione~\ref{sec:literature} vedremo alcuni lavori correlati per quanto riguarda la modellazione in ambito web osservando pregi e difetti del nostro lavoro.\\\\
Nella sezione~\ref{sec:methodology} studieremo il lavoro vero e proprio, introducendo e commentando le scelte architetturali\\\\
Nella sezione~\ref{sec:results} osserveremo il risultato delle scelte metodologiche effettuate, commentando gli sviluppi futuri del progetto\\\\