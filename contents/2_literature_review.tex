\section{Literature review}\label{sec:literature}

As we have already seen in the previous section, this work introduce a platform for building design in the web. We will avoid comparisons with ``desktop apps'', because they are usually complex (and with poor portability), anyway there are many related projects built as web platforms. Some offers only visualization, as the \textbf{Shapespark}\footnote{https://www.shapespark.com/} web viewer (it exists a desktop plugin for modeling), others also offer a modeling service like \textbf{bak3d} (see~\cite{Spini:2016:WIA:2945292.2945309}). On the other hand, bak3d suffer of a complex user interaction. Another complex software is \textbf{Playcanvas}\footnote{https://playcanvas.com/} that is a game engine which offers an integrated physical engine and other functionalities for the modeling results. In the architectural field, another interesting project is \textbf{Floorplan}\footnote{http://www.homestyler.com/floorplan/}, developed by Autodesk.\\
However, our objective was also to explore the web components paradigm and the uniflow pattern. The expected result, is an infrastructure for the building design and an extensible platform for users with the introduction of new components. This is the main difference with the above softwares, because our architecture is not only a modeling instrument but a builder for modeling softwares customized following the users needs.