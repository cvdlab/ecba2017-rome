\section{Results}


E' possibile identificare uno stato centralizzato e consistente che descrive lo stato del progetto\\
Grazie all'algoritmo del virtual dom il 2d viene manutenuto allineata sulla base dello stato in modo automatico\\
Grazie all'algoritmo di diff e patch il 3d viene mutuato senza effettuare un rendering completo, ma applicando le modifiche dello stato\\
Il sistema di lock su layer permette la modifica contemporanea su parti diverse del progetto\\
Servizio di database realtime permette di sincronizzare i vari client sulle modifiche anche in modo concorrente\\
Definizione dei plugin \`e possibile estendere gli elementi geometrici che arricchiscono il progetto architettonico\\
Con l'utilizzo dei componenti \`e possibile comporre l'interfaccia ed esporre il servizio in modo modulare\\



\subsection{Future work}
Il sistema pu\`o essere ampliato grazie alla modularit\`a con cui \`e stato creato. Se dal punto di vista delle funzionalit\`a si riesce ad avere un esperienza utente in grado di accontentare la maggior parte delle esigenze, per aumentare effettivamente i contesti in cui questo software pu\`o essere utilizzato \`e necessario agire sull'integrazione del catalogo attraverso l'aggiunta di ulteriori elementi architettonici.