\section{Introduction}

Scopo di questo lavoro \`e quello di presentare uno strumento di progettazione architettonica basato su web. Dal punto di vista architetturale, è stata introdotta una serverless architecture (vedi citazione [1]). Un'architettura di questo tipo si basa massivamente su servizi di terze parti (typically in the cloud) o su funzioni invocate all'interno di ephemeral containers (may only last for one invocation) per la gestione dello stato e della logica server-side. Questa scelta ha notevoli vantaggi che verranno di seguito presentati.

Questo strumento di progettazione architettonica si propone di realizzare modelli di edifici attraverso l'introduzione di un'interfaccia semplificata che mira ad evitare complicate interazioni con i modelli tridimensionali cercando invece di rimpiazzarle con interazioni su rappresentazioni bidimensionali simboliche dell'edificio.

La natura web ha permesso di sperimentare API per la collaborazione tra utenti remoti, che possono cos\`i progettare un edificio contemporaneamente su pc diversi.

Molta cura \`e stata poi posta sull'espandibilit\`a del sistema, dando agli sviluppatori un elevato livello di personalizzazione. Ciò è possibile: (i) Sfruttando la tecnolgia dei WebComponent per aggiungere nuove funzionalità; (ii) introducendo un \textbf{catalogo} personalizzabile, in grado di fornire elementi architetturali specifici del contesto in cui l'applicazione viene utilizzata; (iii) introducendo un gestore dinamico di metadati, in grado di attribuire informazioni su ogni componente dell'edificio.


Le sezioni che seguono mostrano nel dettaglio i concetti espressi qui, mostrandone la realizzazione e ponendo l'accento sugli aspetti architetturali. (UNA VOLTA COMPLETATO METTERE UN ELENCO PUNTUALE DEL CONTENUTO DELLE SEZIONI)
