\section{Introduction}\label{sec:introduction}

Nowadays we are seeing a relentless migration of software products toward services accessible via the Web medium. This is mainly due to the undeniable benefits in terms of accessibility, usability, maintainability and spreadability granted by the Web medium itself. Nevertheless these benefits don't come without a cost: performance  and development complexity become major concerns in the Web environment. 

In particular, due to the several introduced abstraction layers it is not always feasible to "port" a desktop application into the Web realm, an aspect to be taken into account even for the relevant hardware differences among all the devices equipped with a Web Browser. It can be even more arduous to tackle the inherent distributed software architecture (a client/server one at least) induced by the Web platform. Nevertheless increasingly rich and complex Web applications began to appear, supported by the enriched HTML5 APIs, which thanks to the WebGL~\cite{webgl} (which enables direct access to GPU), Canvas~\cite{Munro:15:HCC} (2D raster APIs) and SVG~\cite{Jackson:11:SVG} (vectorial drawing APIs), has paved the way for the entrance of Web Graphic Applications.

In this work we report about our endeavor toward the definition of a Web based buildings modeling tool which overcomes the aforementioned performance and development difficulties relying on a unidirectional data flow design pattern and on a serverless architecture~\cite{Roberts}, respectively.

A serverless architecture, on the contrary of what the name may suggest, actually employs many different specific servers, whose operation and maintenance don't burden on the project developer(s). These several servers can be seen as third party services (typically cloud-based) or functions executed into ephemeral containers (may only last for one invocation) to manage the internal state and server-side logic. Realtime interaction among users jointly working on the same modeling project, is for example achieved via a third party APIs for remote users collaboration.

The tool user interface, entirely based on web components pattern,  has been kept as simple as possible: the user is required to interact mainly with two-dimensional symbolic placeholders representing parts of the building, thus avoiding complex 3D interactions. 
The modeling complexity is thus moved from the modeler to the developer which fills out an extendible \emph{catalog} of customizable \emph{building elements}. The modeler has only to select the required element, place and parametrize it according to the requirements. It is obvious that a large number of building elements has to be provided to ensure the fulfillment of the most modeling requirements.

The remainder of this document is organized as follows. Section~\ref{sec:related_work} provides an overview of related work. Section~\ref{sec:application} reports about the application user experience. Section~\ref{sec:architecture} presents adopted architectural solutions. Finally, Section ~\ref{sec:conclusions} contains some conclusive remarks.

%%%%%%%%%%%%%%%%%%%%%%%%%%%%%%%%%%%%%%
\iffalse
Scopo di questo lavoro \`e quello di presentare uno strumento di progettazione architettonica basato su web. Dal punto di vista architetturale, \`e stata introdotta una serverless architecture (vedi~\cite{Roberts}). Un'architettura di questo tipo si basa massivamente su servizi di terze parti (typically in the cloud) o su funzioni invocate all'interno di ephemeral containers (may only last for one invocation) per la gestione dello stato e della logica server-side.

Questo strumento si propone di realizzare modelli di edifici attraverso l'introduzione di un'interfaccia semplificata che mira ad evitare complicate interazioni con i modelli tridimensionali cercando invece di rimpiazzarle con interazioni su rappresentazioni bidimensionali simboliche dell'edificio.

La natura web ha permesso di sperimentare API per la collaborazione tra utenti remoti, che possono cos\`i progettare un edificio contemporaneamente su pc diversi.

Molta cura \`e stata poi posta sull'espandibilit\`a del sistema, dando agli sviluppatori un elevato livello di personalizzazione. Ci\`o \`e possibile: (i) Sfruttando web components pattern per aggiungere nuove funzionalità; (ii) introducendo un catalogo personalizzabile, in grado di fornire elementi architetturali specifici del contesto in cui l'applicazione viene utilizzata; (iii) introducendo un gestore dinamico di \textbf{metadati}, in grado di 
attribuire informazioni su ogni componente dell'edificio.

Le sezioni che seguono mostrano nel dettaglio i concetti espressi qui, mostrandone la realizzazione e ponendo l'accento sugli aspetti architetturali. In particolare:\\\\
Nella sezione~\ref{sec:related_work} vedremo alcuni lavori correlati per quanto riguarda la modellazione in ambito web osservando pregi e difetti del nostro lavoro.\\\\
Nella sezione~\ref{sec:methodology} studieremo il lavoro vero e proprio, introducendo e commentando le scelte architetturali\\\\
Nella sezione~\ref{sec:results} osserveremo il risultato delle scelte metodologiche effettuate, commentando gli sviluppi futuri del progetto\\\\

\fi