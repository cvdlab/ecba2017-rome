\section{Application Experience}\label{sec:application}

The application experience focus on supporting the user in a building modeling task. The exploited modeling approach requires the user to face as much as possible a bidimensional interface which allows her to define the floorplan and to place complex architectural elements (here called \emph{building elements}) on it. Such \emph{building elements} can be found in a pre-filled catalog, and when required can be further configured and customized through a side panel. This modeling approach move part of the complexity toward the developer of the customizable building elements, leaving to the final user the task to place and to configure the employed elements. A rich catalog of elements is thus crucial to answer to the users' modeling requirements.

Once the floorplan has been defined according to the \emph{place-and-configure} approach, the system can automatically generate a 3D model which can be explored externally or in first person view, as shown in Figure~\ref{FIGURA_IN_PRIMA_PERSONA}. Each  \emph{building element} in fact comprises either a \emph{2D generating function} than a \emph{3D generating function}, used to obtain models respectively used in the 2D floorplan definition and in 3D generated model.

The tool also has support for layers the user can exploit to organize her project, for example to group together semantically homogenous elements.

\subsection{Building Elements Catalog}\label{ssec:elements}

The catalog comprises four types of elements, grouped according to both inherent characteristics and user interaction needed to add it to the project and configure it.

\begin{itemize}
\item \emph{Lines}. Each line is drawn by selection of a start point and of an end point and we can have two ways to move this type of building element. The first one is by dragging one of the points, the other one is by dragging the entire line. An example: a \emph{wall}.

\item \emph{Openings}. Each opening is an element that is linked to an element of the \emph{line} type, making an hole on it. The user create a new opening dragging it on a chosen line. Examples are \emph{doors} and \emph{windows}.

\item \emph{Areas}. Each area is an element which is automatically generated from walls. An example is the \emph{basement}.

\item \emph{Objects}. Each element which is freely inserted into space with a drag and drop interaction is an object
\end{itemize}

\subsection{User Interface}\label{ssec:ui}

Figure~\ref{FIGURA DELL'INTERFACCIA} show the application user interface. It consist of ...

