\begin{abstract}
The motivations of relentless migration of software products toward services accessible via Web must be sought in the undeniable benefits in terms of accessibility, usability, maintainability and spreadability granted by the Web medium itself. It is the case of office suites, beforehand thought as resilient desktop applications, nowadays made available as Web applications, often equipped with real-time collaboration features and with no need for the user to explicitly install or upgrade them anymore. Although it could not be easy to envisage a Web-based graphic application due to its inherent complexity, after the recent and significant enrichment of the HTML5 APIs, a few first attempts appeared online in the form of vectorial drawing collaborative editors or VR oriented interior design environments.
This paper introduces an effective Web architecture for buildings modeling that leverages the serverless pattern to dominate the developing complexity. The resulting front-end application, powered by Web Components and based on unidirectional data flow pattern, is extremely customizable and extendible by means the definition of plugins to augment the UI or the application functionalities. As regards the modeling approach, it offers (a) to model the building drawing the 2D plans and to navigate the building in a 3D first person point of view; (b) to collaborate in real-time, allowing to work simultaneously on different layers of the project; (c) to define and use new building elements, that are furnitures or architectural components (such as stairs, roofs, etc.), augmenting a ready to use catalog. This work suggests a path for the next-coming BIM online services, matching the professionals collaboration requirements typical of the BIM approach with the platform which supports them the most: the Web.
\end{abstract}