\section{Literature review}

In commercio esistono molti software per la realizzazione di edifici. Si pu\`o pensare ai classici prodotti CAD quali AutoCAD o veri e propri BIM come Revit della Autodesk. Questi sono sicuramente i pi\`u diffusi e completi, tuttavia essendo “desktop apps”, dal punto di vista dell'utente la loro installazione pu\`o essere difficoltosa e gli aggiornamenti pi\`u complicati. Il nostro obiettivo era invece quello di creare una piattaforma semplificata che potesse invece raggiungere praticamente la totalit\`a degli utenti anche su piattaforme differenti. Anche qui vi sono degli esempi, basti pensare al software floorplan di autodesk (http://www.homestyler.com/floorplan/) o anche a strumenti di modellazione tridimensionale vera e propria (Bak3d).

https://www.shapespark.com

Uno dei contributi di questo lavoro consiste nell'esplorazione del nuovo standard dei Web Components e del pattern uniflow. Il risultato atteso \`e quello di riuscire a realizzare non solo una completa infrastruttura per il disegno ma anche una piattaforma estendibile da parte degli utenti mediante l'introduzione di nuovi componenti. In effetti questa \`e la vera e propria differenza con gli strumenti citati precedentemente, perché l'architettura proposta non si limita a risolvere il problema della modellazione ma offre uno strumento di produzione di modellatori architettonici.


AGGIUNGI CITAZIONI ED ESPANDI PARTE SUI WEB COMPONENTS