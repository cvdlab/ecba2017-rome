\section{Introduction}

Scopo di questo lavoro \`e quello di presentare uno strumento di progettazione architettonica basato su web. Dal punto di vista architetturale, \`e stata introdotta una serverless architecture (vedi~\cite{Roberts}). Un'architettura di questo tipo si basa massivamente su servizi di terze parti (typically in the cloud) o su funzioni invocate all'interno di ephemeral containers (may only last for one invocation) per la gestione dello stato e della logica server-side.

Questo strumento si propone di realizzare modelli di edifici attraverso l'introduzione di un'interfaccia semplificata che mira ad evitare complicate interazioni con i modelli tridimensionali cercando invece di rimpiazzarle con interazioni su rappresentazioni bidimensionali simboliche dell'edificio.

La natura web ha permesso di sperimentare API per la collaborazione tra utenti remoti, che possono cos\`i progettare un edificio contemporaneamente su pc diversi.

Molta cura \`e stata poi posta sull'espandibilit\`a del sistema, dando agli sviluppatori un elevato livello di personalizzazione. Ci\`o \`e possibile: (i) Sfruttando \textbf{web components pattern} per aggiungere nuove funzionalità; (ii) introducendo un \textbf{catalogo} personalizzabile, in grado di fornire elementi architetturali specifici del contesto in cui l'applicazione viene utilizzata; (iii) introducendo un gestore dinamico di \textbf{metadati}, in grado di 
attribuire informazioni su ogni componente dell'edificio.

Le sezioni che seguono mostrano nel dettaglio i concetti espressi qui, mostrandone la realizzazione e ponendo l'accento sugli aspetti architetturali. In particolare:\\\\
Nella sezione~\ref{sec:literature} vedremo alcuni lavori correlati per quanto riguarda la modellazione in ambito web osservando pregi e difetti del nostro lavoro.\\\\
Nella sezione~\ref{sec:methodology} studieremo il lavoro vero e proprio, introducendo e commentando le scelte architetturali\\\\
Nella sezione~\ref{sec:results} osserveremo il risultato delle scelte metodologiche effettuate, commentando gli sviluppi futuri del progetto\\\\