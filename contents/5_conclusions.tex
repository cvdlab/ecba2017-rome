\section{Conclusions}\label{sec:conclusions}

In this work we outlined a serverless architecture to support buildings modeling in a Web environment. The serverless  architecture that gives benefits in terms of availability, reliability, scalability, easiness of deployment, maintainability and upgradability is obtained by implementing the application logic as a client-side only centralized state Web application exploiting the unidirectional data flow pattern. This approach allow for a easy-to-serialize state (in the form of a JSON document) that can be pushed on a third party document oriented DB-as-a-Service and loaded back in the frontend reactive architecture, which transparently reload the state once its serialized version is passed in. The application itself is served by a CDN (Content Delivery Network) thus avoiding any need for web server. Offline routines rely on Function-as-a-Service platform as well as users management and collaboration features.

This architecture has been successfully employed by the authors in the \emph{Metior} project~\cite{grapp17}, a tool to support selective deconstruction of buildings in the pursuit of a ``zero waste'' model.
