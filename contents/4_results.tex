\section{Results}\label{sec:results}

Veniamo ora a quali risultati ci ha permesso l'adozione delle tecniche viste prima nello sviluppo dell'applicazione. Il pattern uniflow ci ha permesso di identificare uno stato centralizzato e consistente che descrive lo stato del progetto. In questo modo abbiamo isolato la logica di business e creato un'architettura per la collaborazione poco complessa. Il pattern uniflow associato al virtual dom, ci consente poi di renderizzare i cambiamenti in modo automatico senza complicati algoritmi di ottimizzazione, come il sistema di diff e patch necessario per renderizzare il modello tridimensionale in maniera efficiente. L'approccio serverless e l'introduzione dei layer hanno poi permesso di integrare una piattaforma di collaborazione remota senza grandi sforzi. Inoltre il sistema è espandibile dal punto di vista degli elementi geometrici, dove l'utente pu\`o registrare addirittura dei servizi esterni per servire i propri. Infine, l'utilizzo dei componenti permette di esporre il servizio in modo modulare migliorandone la manutenibilit\`a



\subsection{Future work}
Il sistema pu\`o essere ampliato grazie alla modularit\`a con cui \`e stato creato. Se dal punto di vista delle funzionalit\`a si riesce ad avere un esperienza utente in grado di accontentare la maggior parte delle esigenze, per aumentare effettivamente i contesti in cui questo software pu\`o essere utilizzato \`e necessario agire sull'integrazione del catalogo attraverso l'aggiunta di ulteriori elementi architettonici.