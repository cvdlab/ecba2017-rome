\section{Literature review}

In commercio esistono molti software per la realizzazione di edifici. Si può pensare ai classici prodotti CAD quali AutoCAD o veri e propri BIM come Revit della Autodesk. Questi sono sicuramente i più diffusi e completi, tuttavia essendo “desktop apps”, dal punto di vista dell’utente la loro installazione può essere difficoltosa e gli aggiornamenti più complicati. Il nostro obiettivo era invece quello di creare una piattaforma semplificata che potesse invece raggiungere praticamente la totalità degli utenti anche su piattaforme differenti. Anche qui vi sono degli esempi, basti pensare al software floorplan di autodesk (http://www.homestyler.com/floorplan/) o anche a strumenti di modellazione tridimensionale vera e propria (Bak3d). 

https://www.shapespark.com

Uno dei contributi di questo lavoro consiste nell’esplorazione del nuovo standard dei Web Components e del pattern uniflow. Il risultato atteso è quello di riuscire a realizzare non solo una completa infrastruttura per il disegno ma anche una piattaforma estendibile da parte degli utenti mediante l’introduzione di nuovi componenti. In effetti questa è la vera e propria differenza con gli strumenti citati precedentemente, perché l’architettura proposta non si limita a risolvere il problema della modellazione ma offre uno strumento di produzione di modellatori architettonici.


AGGIUNGI CITAZIONI ED ESPANDI PARTE SUI WEB COMPONENTS